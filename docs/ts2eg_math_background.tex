
\documentclass[11pt]{article}
\usepackage[margin=1in]{geometry}
\usepackage{amsmath,amssymb,amsthm,mathtools,bm}
\usepackage{enumitem}
\usepackage{booktabs}
\usepackage{hyperref}
\hypersetup{colorlinks=true,linkcolor=blue,urlcolor=blue,citecolor=blue}

\title{\textbf{timeseries-to-egt (ts2eg):\\
Mathematical Background and Construction}}
\author{}
\date{\today}

\newtheorem{definition}{Definition}
\newtheorem{proposition}{Proposition}
\newtheorem{remark}{Remark}

\newcommand{\one}{\mathbf{1}}
\newcommand{\R}{\mathbb{R}}
\newcommand{\E}{\mathbb{E}}
\DeclareMathOperator*{\argmin}{arg\,min}

\begin{document}
\maketitle

\section{Overview}
This note collects the mathematics underlying the \emph{timeseries-to-egt} pipeline. The procedure
\begin{enumerate}[leftmargin=*]
    \item constructs per-time \emph{payoff vectors} $v(t)\in\R^N$ from an $N\times T$ multivariate time series $X$,
    \item decomposes $v(t)$ into \emph{common-interest} and \emph{zero-sum} parts via orthogonal projections,
    \item maps player space to a $k$-strategy space using a data-driven dictionary $S\in\R^{N\times k}$ and mixtures $x(t)\in\Delta_k$,
    \item estimates a strategy-level operator $A\in\R^{k\times k}$ compatible with replicator dynamics, and
    \item computes (and tests stability of) mixed equilibria, using IAAFT surrogates to assess significance.
\end{enumerate}
Finding an ESS is interpreted as evidence for \emph{strategic interaction} in the data.

\section{Notation}
\begin{itemize}[leftmargin=*]
\item $X=[X_{\cdot,1}\;\cdots\;X_{\cdot,T}] \in \R^{N\times T}$: $N$ player signals over $T$ times.
\item $\one=(1,\dots,1)^\top\in\R^N$.
\item $\Delta_k = \{x\in\R^k_{\ge 0}:\;\one^\top x = 1\}$: simplex.
\item $M_I, M_Z$: common-interest and zero-sum projectors in player space.
\item $M^{(k)}_Z$: zero-sum projector in strategy space.
\item $S\in\R^{N\times k}$: strategy archetypes (columns).
\item $x(t)\in\Delta_k$: mixture at time $t$; $X_k = [x(1)\,\cdots\,x(T)]\in\R^{k\times T}$.
\item $g(t)\in\R^k$: strategy-level signal at time $t$; $G=[g(1)\,\cdots\,g(T)]$.
\end{itemize}

\section{Common-interest and Zero-sum Subspaces}
\subsection{Unweighted projectors}
Let $\bar v = \tfrac{1}{N}\one^\top v$ denote the arithmetic mean. Define
\begin{equation}
\label{eq:unweighted-projectors}
M_I \;=\; \frac{1}{N}\,\one\one^\top,\qquad
M_Z \;=\; I - M_I.
\end{equation}
For any $v\in\R^N$, $v_I=M_Iv=\bar v\,\one$ is the \emph{common-interest} component and $v_Z=M_Zv=v-\bar v\,\one$ is the \emph{zero-sum} (mean-centered) component. Both $M_I$ and $M_Z$ are symmetric idempotent matrices and satisfy $M_I M_Z=0$, with $\operatorname{im}(M_I)=\operatorname{span}\{\one\}$ and $\operatorname{im}(M_Z)=\{\one\}^\perp$.

\subsection{Weighted projectors}
Let $w\in\R^N_{>0}$ and $D_w=\mathrm{diag}(w)$; define the weighted inner product $\langle u,v\rangle_w = u^\top D_w v$. The orthogonal projector onto $\operatorname{span}\{\one\}$ \emph{with respect to} $\langle\cdot,\cdot\rangle_w$ is
\begin{equation}
\label{eq:weighted-projectors}
M_I^{(w)} \;=\; \one\,\pi_w^\top,\qquad
\pi_w^\top \;=\; \frac{\one^\top D_w}{\one^\top D_w \one} \;=\; \frac{w^\top}{\sum_i w_i},
\end{equation}
so that $M_I^{(w)} v = (\pi_w^\top v)\,\one$ equals the vector with each entry the $w$-weighted mean of $v$. The weighted zero-sum projector is $M_Z^{(w)} = I - M_I^{(w)}$. One checks $D_w M_I^{(w)}=(M_I^{(w)})^\top D_w$ and likewise for $M_Z^{(w)}$, i.e., self-adjointness under $\langle\cdot,\cdot\rangle_w$.

\subsection{Explicit Helmert (contrast) basis}
A particularly convenient orthonormal change of variables is the \emph{Helmert} matrix $Q\in\R^{N\times N}$:
\begin{align}
&\text{First column: } q_1 \;=\; \frac{1}{\sqrt N}\,\one. \label{eq:helmert-q1}\\[3pt]
&\text{For } k=2,\dots,N:\
q_k \text{ has entries } \big(q_k\big)_i \;=\;
\begin{cases}
\frac{1}{\sqrt{k(k-1)}} & \text{if } i<k,\\[3pt]
-\frac{k-1}{\sqrt{k(k-1)}} & \text{if } i=k,\\[3pt]
0 & \text{if } i>k.
\end{cases}
\label{eq:helmert-qk}
\end{align}
Then $Q^\top Q=I$, $Q^\top \one = \sqrt{N}\,e_1$, and
\begin{equation}
M_I \;=\; q_1 q_1^\top \;=\; \frac{1}{N}\one\one^\top,\qquad
M_Z \;=\; I - q_1 q_1^\top \;=\; \sum_{k=2}^N q_k q_k^\top.
\end{equation}
A weighted analogue is obtained by Gram–Schmidt using the $D_w$ inner product, with first vector $q^{(w)}_1=\one/\|\one\|_w$ where $\|\one\|_w^2=\one^\top D_w \one$.

\paragraph{Worked example ($N=3$).}
Using \eqref{eq:helmert-q1}–\eqref{eq:helmert-qk},
\begin{equation}
Q_3 \;=\;
\begin{bmatrix}
\frac{1}{\sqrt{3}} & \frac{1}{\sqrt{2}} & \frac{1}{\sqrt{6}} \\[5pt]
\frac{1}{\sqrt{3}} & -\frac{1}{\sqrt{2}} & \frac{1}{\sqrt{6}} \\[5pt]
\frac{1}{\sqrt{3}} & 0 & -\frac{2}{\sqrt{6}}
\end{bmatrix},\qquad
M_I \;=\; \frac{1}{3}
\begin{bmatrix}
1 & 1 & 1 \\ 1 & 1 & 1 \\ 1 & 1 & 1
\end{bmatrix},\qquad
M_Z \;=\; I - M_I.
\end{equation}
For $v=(p_1,p_2,p_3)^\top$,
\begin{align}
\tilde v \;=\; Q_3^\top v
&= \begin{bmatrix}
\frac{p_1+p_2+p_3}{\sqrt{3}}\\[3pt]
\frac{p_1-p_2}{\sqrt{2}}\\[3pt]
\frac{p_1+p_2-2p_3}{\sqrt{6}}
\end{bmatrix},\qquad
v_I \;=\; \frac{p_1+p_2+p_3}{3}\,\one,\qquad
v_Z \;=\; v - v_I.
\end{align}

\section{From Time Series to Payoffs}
\subsection{Static (profile) game}
Let $Z=M_Z X$ (or $M_Z^{(w)}X$). Discretize each row of $Z$ into $b$ bins to form a joint \emph{profile} $a(t)\in\{1,\dots,b\}^N$. For each profile $a$, define a payoff vector
\begin{equation}
v(a) \;=\; \E\!\big[X_{\cdot,t+1}\mid a(t)=a\big],
\end{equation}
estimated empirically. At time $t$ set $v(t)=v\!\big(a(t)\big)$ and take $v_Z(t)=M_Z v(t)$ (or weighted).

\subsection{Information-sharing (VAR) game}
For each player $i$, consider two linear predictors of $X_{i,t}$ from past lags up to order $p$:
\begin{align}
\text{Self model: }\quad & \hat X_{i,t} = \sum_{\ell=1}^p \alpha_{i,\ell} X_{i,t-\ell} + u_{i,t},\\
\text{Full model: }\quad & \hat X_{i,t} = \sum_{j=1}^N \sum_{\ell=1}^p \beta_{i,j,\ell} X_{j,t-\ell} \;+\; \sum_{m\in\mathcal{S}} \gamma_{i,m} s_m(t) \;+\; e_{i,t},
\end{align}
where $s_m(t)$ are optional seasonal regressors (e.g., $\sin/\cos$ at multiples of a base period). Let $\sigma^2_{i,\mathrm{self}}=\E[u_{i,t}^2]$ and $\sigma^2_{i,\mathrm{full}}=\E[e_{i,t}^2]$. Define the \emph{information-sharing payoff} for player $i$ as the reduction in mean squared error
\begin{equation}
v_i \;:=\; \sigma^2_{i,\mathrm{self}} - \sigma^2_{i,\mathrm{full}} \quad (\text{optionally normalized}).
\end{equation}
Finally set $v_Z = M_Z v$ (or $M_Z^{(w)}v$).

\section{Strategies and Mixtures}
Let $S=[s_1\,\cdots\,s_k]\in\R^{N\times k}$ be a strategy dictionary (e.g.\ via nonnegative matrix factorization). For each time $t$, infer $x(t)\in\Delta_k$ by projecting the observed signal onto the cone spanned by $S$ with a simplex constraint, e.g.
\begin{equation}
x(t)\;\in\;\argmin_{x\in\Delta_k}\;\|X_{\cdot,t}-Sx\|_2^2\quad (\text{optionally with nonnegativity on }S).
\end{equation}
Define the strategy-level signal $g(t)=S^\top v_Z(t)$ and collect $X_k=[x(1)\,\cdots\,x(T)]\in\R^{k\times T}$, $G=[g(1)\,\cdots\,g(T)]\in\R^{k\times T}$.

\section{Replicator-compatible Operator}
Let $M_Z^{(k)}=I_k-\tfrac{1}{k}\one\one^\top$ be the strategy-space centering matrix. We fit $A\in\R^{k\times k}$ via ridge regression:
\begin{equation}
\label{eq:ridge-A}
\min_A \;\big\|M_Z^{(k)}G - A X_k\big\|_F^2 + \lambda\|A\|_F^2
\qquad\Longrightarrow\qquad
A \;=\; \big(M_Z^{(k)}G\big)X_k^\top\;\big(X_k X_k^\top + \lambda I\big)^{-1}.
\end{equation}
Row-centering $A\one=0$ (and optionally $\one^\top A=0$) enforces invariance of replicator dynamics under constant payoff shifts.

\section{Replicator Dynamics, Jacobian, Nash, and ESS}
The continuous-time replicator dynamics on $\Delta_k$ are
\begin{equation}
\dot x_i \;=\; x_i \Big((Ax)_i - \bar f(x)\Big),\qquad \bar f(x)=x^\top A x,\quad i=1,\dots,k.
\end{equation}
The Jacobian of the vector field $F_i(x)=x_i\big((Ax)_i - x^\top A x\big)$ at an interior point $x$ is
\begin{equation}
\frac{\partial F_i}{\partial x_j}(x) \;=\; \delta_{ij}\,\big((Ax)_i - x^\top A x\big) \;+\; x_i\Big(A_{ij} - (A^\top x)_j - (Ax)_j\Big).
\end{equation}
A mixed rest point $x^\star$ with support $J=\{i:x^\star_i>0\}$ satisfies
\begin{equation}
A_{JJ} x^\star_J = \alpha\,\one,\quad \one^\top x^\star_J=1,\qquad (Ax^\star)_\ell \le \alpha\ \ \forall \ell\notin J,
\end{equation}
i.e., $x^\star$ is a (symmetric) mixed Nash equilibrium. For (local) evolutionary stability one may use the following sufficient test.
\begin{proposition}[Tangent-space test]
If $x^\star$ is a mixed Nash equilibrium with support $J$ and the quadratic form
\(
y^\top \frac{A_{JJ}+A_{JJ}^\top}{2}\, y
\)
is negative definite on $\{y\in\R^{|J|}:\ \one^\top y=0\}$, then $x^\star$ is a (locally) evolutionary stable strategy for the replicator dynamics.
\end{proposition}

\section{Seasonal VAR Details}
Let $S(t)$ collect seasonal regressors such as $\{\sin(2\pi m t/P),\cos(2\pi m t/P)\}_{m\in \mathcal{M}}$. For player $i$ the full regression is
\begin{equation}
X_{i,t} \;=\; \sum_{j=1}^N \sum_{\ell=1}^p \beta_{i,j,\ell} X_{j,t-\ell} \;+\; \gamma_i^\top S(t) \;+\; e_{i,t},
\end{equation}
with ridge penalty on $\beta,\gamma$. The information-sharing payoff $v_i$ is computed from the MSE reduction relative to the self-only regression. Center $v$ across players with $M_Z$ (or $M_Z^{(w)}$). See \cite{lutkepohl} for VAR practice.

\section{IAAFT Surrogates and Significance}
To assess whether an observed ESS arises from cross-player coupling rather than marginal autocorrelation, we generate IAAFT surrogates for each player series:
\begin{itemize}[leftmargin=*]
\item Each surrogate preserves the \emph{empirical amplitude distribution} and the \emph{power spectrum} of the original player series but destroys cross-player dependence.
\item For each surrogate dataset, recompute payoffs, re-estimate $A$ via \eqref{eq:ridge-A}, and tally whether an ESS is found. The surrogate ESS rate provides a null benchmark.
\end{itemize}
IAAFT alternates phase randomization in the Fourier domain (matching the original magnitudes) with rank-order remapping to the original amplitudes until convergence \cite{schreiber-schmitz}.

\section{Edge Cases and Identifiability}
\begin{itemize}[leftmargin=*]
\item If $X_{\cdot,t}=c_t\,\one$ (pure common mode), then $v_Z\equiv 0$ and the EGT signal is null by design.
\item Replicator dynamics are invariant under adding the same constant to all payoffs; centering $A$ enforces this.
\item Choice of $S$ matters; NMF/archetypes produce parts-based, interpretable strategies. PCA is acceptable but mixes signs; rectification may be used.
\end{itemize}

\section{Relation to Jessie \& Saari's Coordinate Systems}
Jessie and Saari advocate analyzing \emph{classes} of games by decomposing each game into coordinates that separate individual incentives (\emph{``Nash''}), group/behavioral components, and kernel (payoff inflation/deflation), yielding a uniform methodology across formats \cite{jessie-saari}. In their words, a coordinate system ``explicitly separates each payoff into contributions being made to each feature,'' replacing ad hoc analysis with a systematic decomposition.\\[4pt]
\emph{How ts2eg differs.} Our construction is \textbf{time-series first}. Rather than starting from a static payoff table and decomposing it, we:
\begin{enumerate}[leftmargin=*]
\item derive per-time payoff vectors from multivariate signals (static or predictive-gain definitions);
\item isolate common-interest vs.\ zero-sum directions \emph{in player space} via ANOVA/Helmert at each time;
\item map payoffs to a learned strategy dictionary $S$ to obtain strategy-level signals; and
\item estimate a replicator-compatible operator $A$ by ridge, enabling ESS analysis with surrogate significance (IAAFT).
\end{enumerate}
Thus, while philosophically aligned with ``coordinate systems for games,'' ts2eg couples signal processing (VAR with seasonal regressors; IAAFT) and evolutionary dynamics through an explicit change of variables in player space.

\section{Algorithms (summary)}
\begin{enumerate}[leftmargin=*]
\item Compute $M_I$/$M_Z$ (or weighted) and optionally a (weighted) Helmert basis using \eqref{eq:helmert-q1}–\eqref{eq:helmert-qk}.
\item Build per-time payoffs $v(t)$ via static profiles or information-sharing; take $v_Z(t)=M_Z v(t)$.
\item Learn strategies $S$ and mixtures $x(t)\in\Delta_k$.
\item Form $g(t)=S^\top v_Z(t)$ and fit $A$ by ridge: $A=(M_Z^{(k)}G)X_k^\top(X_kX_k^\top+\lambda I)^{-1}$; enforce $A\one=0$.
\item Enumerate supports to find mixed Nash equilibria; apply the tangent-space test for ESS.
\item Generate IAAFT surrogates; re-estimate $A$; report ESS frequency.
\end{enumerate}

\section{Symbols (quick reference)}
\begin{center}
\begin{tabular}{ll}
$X\in\R^{N\times T}$ & player signals (rows = players, cols = time) \\
$M_I, M_Z$ & common-interest and zero-sum projectors \\
$M_I^{(w)}, M_Z^{(w)}$ & weighted projectors (inner product $\langle\cdot,\cdot\rangle_w$) \\
$Q$ & (weighted) Helmert matrix (orthonormal change of basis) \\
$S\in\R^{N\times k}$ & strategy archetypes (columns) \\
$x(t)\in\Delta_k$ & mixture on the simplex at time $t$ \\
$g(t)=S^\top v_Z(t)$ & strategy-level signal \\
$A\in\R^{k\times k}$ & replicator-compatible operator \\
\end{tabular}
\end{center}

\begin{thebibliography}{9}
\bibitem{monderer-shapley}
D. Monderer and L. S. Shapley (1996).
\newblock Potential Games.
\newblock \emph{Games and Economic Behavior} 14(1):124--143.

\bibitem{candogan}
O. Candogan, I. Menache, A. Ozdaglar, and P. A. Parrilo (2011).
\newblock Flows and Decompositions of Games: Harmonic and Potential Games.
\newblock \emph{Mathematics of Operations Research} 36(3):474--503.

\bibitem{hofbauer-sigmund}
J. Hofbauer and K. Sigmund (1998).
\newblock \emph{Evolutionary Games and Population Dynamics}.
\newblock Cambridge University Press.

\bibitem{sandholm}
W. H. Sandholm (2010).
\newblock \emph{Population Games and Evolutionary Dynamics}.
\newblock MIT Press.

\bibitem{lutkepohl}
H. L\"{u}tkepohl (2005).
\newblock \emph{New Introduction to Multiple Time Series Analysis}.
\newblock Springer.

\bibitem{schreiber-schmitz}
T. Schreiber and A. Schmitz (2000).
\newblock Surrogate time series.
\newblock \emph{Physica D} 142(3--4):346--382.

\bibitem{jessie-saari}
D. T. Jessie and D. G. Saari (2019).
\newblock \emph{Coordinate Systems for Games: Simplifying the ``me'' and ``we'' Interactions}.
\newblock Springer, in \emph{Static \& Dynamic Game Theory: Foundations \& Applications}.
\end{thebibliography}

\end{document}
